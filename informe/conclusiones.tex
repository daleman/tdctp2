\section{Conclusiones.}

Si bien la tool diseñada y el método propuesto nos permiten obtener la ruta a un website detectando los posibles enlaces intercontinentales, este procedimiento está sujeto a numerosas adversidades. Routers que no responden (aún con timeouts largos) o que asignan prioridad baja a los paquetes ICMP,  geolocalizadores inexactos, congestión en una ruta o rutas distintas para un mismo website son mayormente lo que notamos que afecta considerablemente a los resultados obtenidos.  Veamos algunas de estas problemáticas en detalle:

En los casos en los que no obtenemos una respuesta del router, no tenemos forma de conocer, con este método, esa parte de la ruta, aún sabiendo que allí debería haber un nodo. Esto afecta a la comparativa general ya que, en caso de tener numerosas no-respuestas sucesivas, el RTT relativo entre los extremos de esta secuencia podría distribuirse de diversos modos. Un caso posible consiste en una distribución equitativa de los RTTs entre los nodos de la secuencia significando que ninguno corresponde a un enlace submarino. A su vez podría suceder que la distribución sea desigual y exista un enlace intermedio de elevado RTT representando una conexión intercontinental.

Por otra parte, en los casos en donde obtenemos una respuesta, esta información no necesariamente basta. La estimación de la distancia entre hops mediante el RTT no resulta demasiado confiable debido a que el tiempo de ida y vuelta (de la forma calculada) no depende única y exclusivamente de la longitud del enlace sino también incluye el tiempo del paquete dentro del router que podría extenderse por congestión, prioridad u otros factores. Además la congestión a lo largo de la red puede no ser uniforme a lo largo de la ruta en un momento dado y cada router puede tener una asignación de prioridades diferente. De esta manera la comparación entre RTTs se ve afectada no sólo por el momento de la medición sino tambien por los nodos involucrados en la ruta elegida. Una forma posible de resolver la congestión en un horario podría ser tomar mediciones en distintos momentos del día, pudiendo así tener una idea de cómo afecta la congestión a nuestras mediciones.

A pesar de esto, encontramos casos en los que el método pudo ser llevado a cabo de forma efectiva.
